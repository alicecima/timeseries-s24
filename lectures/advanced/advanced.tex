\documentclass{article}

\def\ParSkip{} 
% Packages
\usepackage{amssymb,amsmath,amsthm,bbm}
\usepackage{verbatim,float,url,dsfont}
\usepackage{graphicx,subfigure,psfrag}
\usepackage{algorithm,algorithmic}
\usepackage{mathtools,enumitem}
\usepackage{multirow}
\usepackage{ragged2e}
\usepackage{xr-hyper}
\usepackage{array}

\usepackage[colorlinks=true,citecolor=blue,urlcolor=blue,linkcolor=blue]{hyperref}
\usepackage[margin=1in]{geometry}
\usepackage[round]{natbib}

\usepackage[utf8]{inputenc} % allow utf-8 input
\usepackage[T1]{fontenc}    % use 8-bit T1 fonts
\usepackage{booktabs}       % professional-quality tables
\usepackage{nicefrac}         % compact symbols for 1/2, etc.
\usepackage{microtype}      % microtypography

\ifdefined\TimesFont 
\usepackage{times} % use times font
\fi

\ifdefined\ParSkip 
\usepackage{parskip} % use par skip
\fi

% Theorems and such
\newtheorem{theorem}{Theorem}
\newtheorem{lemma}{Lemma}
\newtheorem{corollary}{Corollary}
\newtheorem{proposition}{Proposition}
\theoremstyle{definition}
\newtheorem{remark}{Remark}
\newtheorem{definition}{Definition}

% Assumption
\newtheorem*{assumption*}{\assumptionnumber}
\providecommand{\assumptionnumber}{}
\makeatletter
\newenvironment{assumption}[2]{
  \renewcommand{\assumptionnumber}{Assumption #1#2}
  \begin{assumption*}
  \protected@edef\@currentlabel{#1#2}}
{\end{assumption*}}
\makeatother

% Widebar
\makeatletter
\newcommand*\rel@kern[1]{\kern#1\dimexpr\macc@kerna}
\newcommand*\widebar[1]{%
  \begingroup
  \def\mathaccent##1##2{%
    \rel@kern{0.8}%
    \overline{\rel@kern{-0.8}\macc@nucleus\rel@kern{0.2}}%
    \rel@kern{-0.2}%
  }%
  \macc@depth\@ne
  \let\math@bgroup\@empty \let\math@egroup\macc@set@skewchar
  \mathsurround\z@ \frozen@everymath{\mathgroup\macc@group\relax}%
  \macc@set@skewchar\relax
  \let\mathaccentV\macc@nested@a
  \macc@nested@a\relax111{#1}%
  \endgroup
}
\makeatother

% Min and max 
\DeclareMathOperator*{\argmin}{argmin}
\DeclareMathOperator*{\argmax}{argmax}
\DeclareMathOperator*{\minimize}{minimize}
\DeclareMathOperator*{\maximize}{maximize}
\DeclareMathOperator*{\find}{find}
\DeclareMathOperator{\st}{subject\,\,to}

% Other operators
\DeclareMathOperator{\Cov}{Cov}
\DeclareMathOperator{\Var}{Var}
\DeclareMathOperator{\dm}{dim}
\DeclareMathOperator{\col}{col}
\DeclareMathOperator{\row}{row}
\DeclareMathOperator{\nul}{null}
\DeclareMathOperator{\rank}{rank}
\DeclareMathOperator{\nuli}{nullity}
\DeclareMathOperator{\spa}{span}
\DeclareMathOperator{\sign}{sign}
\DeclareMathOperator{\supp}{supp}
\DeclareMathOperator{\diag}{diag}
\DeclareMathOperator{\aff}{aff}
\DeclareMathOperator{\conv}{conv}
\DeclareMathOperator{\dom}{dom}
\DeclareMathOperator{\tr}{tr}
\DeclareMathOperator{\df}{df}

% Other shortcuts 
\def\R{\mathbb{R}}
\def\C{\mathbb{C}}
\def\E{\mathbb{E}}
\def\P{\mathbb{P}}
\def\T{\mathsf{T}}
\def\half{\frac{1}{2}}
\def\df{\mathrm{df}}
\def\hy{\hat{y}}
\def\hf{\hat{f}}
\def\hmu{\hat{\mu}}
\def\halpha{\hat{\alpha}}
\def\hbeta{\hat{\beta}}
\def\htheta{\hat{\theta}}
\def\indep{\perp\!\!\!\perp}
\def\th{^{\textnormal{th}}}

\def\cA{\mathcal{A}}
\def\cB{\mathcal{B}}
\def\cD{\mathcal{D}}
\def\cE{\mathcal{E}}
\def\cF{\mathcal{F}}
\def\cG{\mathcal{G}}
\def\cK{\mathcal{K}}
\def\cH{\mathcal{H}}
\def\cI{\mathcal{I}}
\def\cL{\mathcal{L}}
\def\cM{\mathcal{M}}
\def\cN{\mathcal{N}}
\def\cP{\mathcal{P}}
\def\cS{\mathcal{S}}
\def\cT{\mathcal{T}}
\def\cW{\mathcal{W}}
\def\cX{\mathcal{X}}
\def\cY{\mathcal{Y}}
\def\cZ{\mathcal{Z}}


\title{Lecture 8: Advanced Topics: Advanced Forecasting Methods, Calibration,
  Scoring, and Ensembling \\ \smallskip
\large Introduction to Time Series, Fall 2023 \\ \smallskip
Ryan Tibshirani}
\date{}

\begin{document}
\maketitle
\RaggedRight
\vspace{-50pt}

\section{Advanced forecasters}

\begin{itemize}
\item Thus far, we've learned in-depth about ARIMA and ETS as our two major 
  forecasting frameworks. Undoutedbly, these are batted-tested frameworks that 
  have been used for decades and will take you far, albeit with proper scrutiny 

\item Of course, there are actually many other forecasting methods out there,
  and this continues to be an active topic of research. Below, we briefly
  summarize three other forecasting methods, chosen based on their popularity
  (they also seem to constitute a common cast of characters, together with ARIMA
  and ETS, in R and Python forecasting packages)       
\end{itemize}

\subsection{Theta model}

\begin{itemize}
\item First on our list is the \emph{Theta model}, proposed by
  \citet{assimakopoulos2000theta}. This is on our list not as an advanced
  forecaster per se (as you will see, it's actually quite simple)

\item Instead, it is a simple and popular method that began gaining a lot of
  attention in parts of the forecasting community, and is closely connected to
  something you already know: exponential smoothing    

\item Our presentation here follows \citet{hyndman2003unmasking}, who give a
  nice, clear perspective on the Theta method and its connection to exponential
  smoothing  

\item Given data $x_t$, $t = 1,2,3,\dots$, the Theta method starts by defining a
  smoothed sequence $y_{\theta,t}$, $t = 1,2,3,\dots$ by the second-order
  difference equation:
  \[
  \Delta^2 y_{\theta,t} = \theta \Delta^2 x_t 
  \]
  Here $\Delta^2 = (1-B)^2$ is the second-order difference operator, just like
  in our ARIMA lecture, and $\theta \geq 0$ is a parameter

\item The solution to the above is given by 
  \[
  y_{\theta,t} = a_\theta + b_\theta t + \theta x_t
  \]
  for some (constant in time) intercept and slope parameters $a_\theta,
  b_\theta$  

\item For fixed $\theta$, the intercept and slope parameters are fit by
  minimizing the sum of squared errors to the original sequence  
  \[
  \min_{a_\theta, b_\theta} \, \sum_{t=1}^n (x_t - y_{\theta,t})^2 \iff
  \min_{a_\theta, b_\theta} \, \sum_{t=1}^n \Big( (1-\theta) x_t - a_\theta +
  b_\theta t  \Big)^2 
  \]
  which is simply a linear regression of $(1-\theta) x_t$ on time $t$ 

\item Once these estimates \smash{$\hat{a}_\theta, \hat{b}_\theta$} are found,
  forecasts are made by running simple exponential smoothing (SES) on the
  sequence  
  \[
  \hat{y}_{\theta,t} = \hat{a}_\theta + \hat{b}_\theta t + \theta x_t
  \]
  if $\theta > 0$, or by extrapolating the line \smash{$\hat{a}_0 +
    \hat{b}_0 t$} forward in time if $\theta = 0$

\item \citet{assimakopoulos2000theta} make the following general recommendation:
\begin{itemize}
\item produce forecasts \smash{$\hat{y}_{0,t+h|t}$} with $\theta = 0$ (recall
  this is just extending the line \smash{$\hat{a}_0 + \hat{b}_0 t$});
\item produce forecasts \smash{$\hat{y}_{2,t+h|t}$} with $\theta = 2$ (recall
  this is given by just running SES on \smash{$\hat{y}_{2,t}$}) 
\item return their average: \smash{$(\hat{y}_{0,t+h|t} + \hat{y}_{2,t+h|t}) / 2$}.
\end{itemize}
Seasonality, if present, is estimated and removed before running this procedure,
and added back in at the end

\item \citet{hyndman2003unmasking} show that this procedure is quite similar to
  running Holt's linear trend method with an estimated slope \smash{$\hat{b}_0 /
    2$} and with $\beta = 0$ (no evolution of the slope over time) 

\item They also compare the Theta method with Holt's linear trend method on the
  daa from the M3 forecasting challenge (where the Theta method performed well
  and subsequently gained popularity), and observe that Holt's linear trend
  performs competitively  

\item It is worth knowing about the close connections between Theta and
  exponential smoothing, because much of the literature on the Theta model does
  not seem to emphasize this aspect. There has been more recent work on the
  Theta model (which may further distinguish it from exponential smoothing---we
  cannot say, because we have not followed it) that you may be interested in
  reading    
\end{itemize}

\subsection{Prophet model}

\begin{itemize}
\item Next on our list is the \emph{Prophet model}, proposed by
  \citet{taylor2018forecasting}, from Facebook. This has become popular for
  large-scale forecasting enterprises, and the popular opinion seems to be that
  its advantages over traditional ARIMA or ETS models are twofold: flexibility
  and speed. But make sure to read on, especially to the end of this subsection,
  for further discussion of this 

\item The Prophet model is a particular type of signal plus noise model, where
  we model the given time series as
  \[
  x_t = g_t + s_t + h_t + \epsilon_t
  \]
  where $\epsilon_t$, $t = 1,2,3,\dots$ is a white noise sequence, and:
  \begin{itemize}
  \item $g_t$ represents a trend component
  \item $s_t$ represents a seasonal component
  \item $h_t$ captures holiday/calendar effects 
  \end{itemize}

\item This can be seen as a particular type of smoother, based on a particular
  model for the trend (which we will describe shortly; the seasonal and holiday
  components are fairly generic). We could also refer to it as a particular type
  of additive model, where the regressor is time

\item The seasonal component is parametrized by a Fourier (cosine and sine)
  basis at given fixed, known periods chosen by the user. For frequencies
  $\omega_j$, $j = 1,\dots,p$ (equivalently, periods $1/\omega_j$, $j =
  1,\dots,p$), recall, this is:  
  \[
  g_t = \sum_{j=1}^p \Big( a_j \cos(2\pi \omega_j t) + b_j \sin(2\pi \omega_j t)
  \Big) 
  \]
  for coefficients $a_j, b_j$, $j = 1,\dots,p$

\item The holiday/calendar component is simply parametrized using indicator
  variables 
  \[
  h_t = \sum_{j=1}^m \alpha_j \cdot 1\{ t \in D_j \}
  \]
  where $\alpha_j$, $j = 1,\dots,m$ are coefficients and each $D_j$ is a set of
  dates representing a particular holiday or calendar event (e.g., Christmas,
  Thanksgiving, etc.) 

\item Finally, the trend component is modeled in one of two ways. The first way
  is for \emph{saturating} trends. For this, \citet{taylor2018forecasting}
  propose to model $g_t$ using a sigmoid function with a piecewise growth rate:  
  \[
  g_t = \frac{C(t)}{1 + \exp\Big( c_0 + c_1 t + \alpha \sum_{j=1}^r \beta_j
    \cdot (t-t_j)_+\Big)} 
  \]
  Here $C(t)$ is a (possibly) time-varying capacity, which it appears
  \citet{taylor2018forecasting} recommend be set externally (e.g., based on 
  market sizes considerations)

\item For \emph{non-saturating} trends, \citet{taylor2018forecasting}
  propose to model $g_t$ using a piecewise linear trend directly: 
  \[
  g_t = c_0 + c_1 t + \alpha \sum_{j=1}^r \beta_j \cdot (t-t_j)_+ 
  \]

\item In either case (saturating or non-saturating), $\beta_j$, $j = 1,\dots,r$
  are coefficients to be estimated. Also, $t_j$, $j = 1,\dots,r$ are knots
  (where the slope changes), which, in the simplest case, can be fixed ahead of 
  time. Instead, \citet{taylor2018forecasting} recommend that knots be selected
  using $\ell_1$ penalization from a large initial set of locations. That is,
  they use the $\ell_1$ penalty 
  \[
  \sum_{j=1}^r |\beta_j|
  \]
  when fitting the model, which is like a special type of lasso regression. (In
  fact, though it may not be obvious at first pass, placing an $\ell_1$ penalty
  on $\beta_j$, $j = 1,\dots,r$ here is actually equivalent to reparametrizing
  the entire sequence as $g_t = \theta_t$, and then using an $\ell_1$ penalty on  
  second differences of $\theta_t$; recall, this is the penalization scheme used
  in \emph{trend filtering}, which you learned about earlier in the course when
  we covered smoothing)  

\item Altogether, the Prophet model is fit by minimizing the sum of squared
  errors to the observed data, over all parameters
  $a_j,b_j,\alpha_j,c_0,c_1,\beta_j$ that determine the decomposition, with a
  squared $\ell_2$ (ridge) penalty on the parameters $a_j,b_j,\alpha_j$ for the
  seasonal component $s_t$ and holiday component $h_t$, and an $\ell_1$ (lasso)
  penalty on the parameters $\beta_j$ for the trend component $g_t$  

\item Forecasts are generated by extrapolating the fitted components forward in 
  time. For $s_t$ and $h_t$, this is straightforward, because they are periodic
  in nature. For $g_t$, this is done by holding the slope (i.e., growth rate in
  the saturating model) constant from its last value, as we move forward in
  time 

\item (Though unimportant for our purposes here, \citet{taylor2018forecasting}
  actually phrase all of this in the context of a hierarchical Bayesian model,
  with normal and Laplace priors that serve the purpose of regularization;  
  this Bayesian machinery also provides added stochasticity in computation of 
  prediction intervals) 

\item So, how does the Prophet model compare to ARIMA and ETS? It depends on who
  you ask. In their original paper, \citet{taylor2018forecasting} find ARIMA and
  ETS models to be too rigid in their motivating examples---take a look at
  Figure 3 in their paper, and compare Figure 4, which displays Prophet forecasts

\item Indeed, in their traditional forms, ARIMA and ETS models lack the
  flexibility of the Prophet model, particularly the flexibility exhibited in
  the latter's trend component. However, both ARIMA and ETS can be extended to
  accommodate more sophisticated trends. With ARIMA, you have actually already
  seen a way to do this: we can phrase the problem as \emph{regression with
    correlated errors} (where we use an ARIMA model for the errors), and we can
  use time for the regressor and create flexible basis functions to model trends
  just like Prophet does    

\item Hyndman and Athanasopoulos (HA) call this a \emph{dynamic regression
    model} and study it in Chapter 10 of their book. The advantage this has over 
  Prophet is that it is able to capture auto-correlations in the errors, which
  can lead to narrower prediction intervals. The advantage Prophet has is speed:
  it is usually more efficient to fit the Prophet model, since its error model
  (white noise) is simpler and this makes the optimization a version of
  penalized least squares  
\end{itemize}

\subsection{Neural network autoregression}

\begin{itemize}
\item A \emph{neural network} is a class of models that make predictions $f(x)$
  from an input (feature vector) $x \in \R^p$ of the form:
  \begin{align*}
  f_1(x) &= \rho(W_1 x + b_1) \\
  f_\ell(x) &= \rho \Big( W_{\ell-1} f_{\ell-1}(x) + b_{\ell-1} \Big), \quad
              \ell = 2,\dots,L \\
  f(x) &= f_L(x)
  \end{align*}

\item Here each $W_\ell \in \R^{d_\ell \times d_{\ell-1}}$ is a matrix of
  weights that maps from the dimension $d_{\ell-1}$ of layer $\ell-1$ to the
  dimension $d_\ell$ of layer $\ell$. Note that $d_0 = p$, and $d_L = 1$ (for
  real-valued predictions)

\item Each $b_\ell \in \R^{d_\ell}$ is a vector of intercepts (often called
  \emph{biases} in the deep learning community). Generically, the parameters
  $W_\ell,b_\ell$ are all learned by minimizing the sum of squared errors of
  predictions on the training data 

\item The function $\rho$ is a nonlinear \emph{activation function} that is 
  interpreted as being applied componentwise, and user-chosen; common choices
  are $\rho(u) = u_+$ and $\rho(u) = 1/(1+e^{-u})$

\item Lastly, $L$ here is the number of layers, also a design choice, and often 
  called the \emph{depth} of the network   

\item For time series, one of the simplest things that can be done with neural
  networks is just to form input features by taking lags of the given response
  variable. We can use both nonseasonal and seasonal lags (as in ARIMA). This
  gives rise to what we call a \emph{neural network autoregressive} (NNAR) model  

\item What we described above is actually just a particular type of neural
  network architecture, and indeed, the simplest kind, called a
  \emph{feedforward} neural network. Many other architectures are possible, and
  some more appropriate for time series data, such as the \emph{long short-term
    memory} (LSTM) network, a type of recurrent neural network

\item A popular time series forecaster based on LSTMs is called \emph{DeepAR},
  proposed by \citet{salinas2020deep}, from Amazon. Relative to all other
  methods you have learned thus far, DeepAR is quite complicated to describe
  precisely (and to train). However, like many deep learning methods, it can
  work very well in data-rich prediction problems that have a high
  signal-to-noise ratio 

\item Perhaps not surprisingly, some authors are now re-purposing transformers 
  in order to turn them into time series forecasters. As deep learning continues
  to grow, we will continue to see spillover into time series forecasting 
\end{itemize}

\section{Calibration}

\begin{itemize}
\item 
\end{itemize}

\section{Scoring}

\begin{itemize}
\item 
\end{itemize}

\section{Ensembling}

\begin{itemize}
\item 
\end{itemize}

\subsection{Untrained methods}

\begin{itemize}
\item 
\end{itemize}

\subsection{Trained methods}

\begin{itemize}
\item 
\end{itemize}

\bibliographystyle{plainnat}
\bibliography{advanced}

\end{document}
